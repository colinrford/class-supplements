\documentclass[12pt,a4]{article}

\usepackage{amsfonts,amsmath,amssymb,amsthm}
\usepackage{amscd}
\usepackage{mathtools}
\usepackage{geometry, algorithmicx} 
\usepackage[noend]{algpseudocode}
\usepackage{subfig}
\usepackage{graphicx}
\usepackage{dsfont}
\usepackage{fancyhdr}
\usepackage{enumerate}
\usepackage{pgf, tikz}
\usetikzlibrary{shapes,snakes}
\usetikzlibrary{arrows, automata}
\theoremstyle{definition}


\fancyhf{}
\fancyhead[LE,RO]{\thepage}
\fancyhead[CE]{\Author}
\fancyhead[CO]{\Title}
\renewcommand\headrulewidth{0pt}
\pagestyle{fancy}

\author{Colin Ford}
\title{Ascher - Chapter 2 Exercises}
\date{}

\makeatletter
\let\Title\@title
\makeatother

\newtheorem*{theorem*}{Theorem}
\newtheorem*{proposition*}{Proposition}
\newtheorem{problem}{Problem}
\newtheorem*{problem*}{Problem}
\newtheorem{exercise}{Exercise}
\newtheorem{lemma}{Lemma}
\newtheorem*{definition*}{Definition}
\newtheorem*{lemma*}{Lemma}
\newtheorem*{claim*}{Claim}
\newtheorem*{example}{Example}

\begin{document}

\maketitle

% % % % % % % % % % %
% % % Problem 1 % % %
% % % % % % % % % % %
\begin{exercise}[Review Questions]
	\begin{enumerate}[(a)]
		% % a % %
		\item What is the difference, according to Section 1.1, between scientific computing and numerical analysis?
		
		% % b % %
		\item Give a sample example where relative error is a more suitable measure than absolute error, and another example where the absolute error measure is more suitable. 
		
		% % c % %
		\item State a major difference between the nature of roundoff errors and discretization errors.
		
		% % d % %
		\item Explain briefly why accumulation of roundoff errors is inevitable when arithmetic operations are performed in a floating point system. Under which circumstances is it tolerable in numerical computations?
		
		% % e % %
		\item Explain the differences between accuracy, efficiency, and robustness as criteria for evaluating an algorithm.
		
		% % f % %
		\item Show that nested evaluation of a polynomial of degree $n$ requires only $2n$ elementary operations and hence has $O(n)$ complexity.
		
		% % g % %
		\item Distinguish between problem conditioning and algorithm stability.
		
	\end{enumerate}
\end{exercise}
\begin{proof}[Solution]
	\begin{enumerate}[(a)]
		% % a % %
		\item 
		
		% % b % %
		\item 
		
		% % c % %
		\item 
		
		% % d % %
		\item 
		
		% % e % %
		\item 
		
		% % f % %
		\item 
		
		% % g % %
		\item 
		
	\end{enumerate}
\end{proof}

% % % % % % % % % % %
% % % Problem 2 % % %
% % % % % % % % % % %
\begin{exercise}
	 
\end{exercise}
\begin{proof}[Solution]
	
\end{proof}

% % % % % % % % % % %
% % % Problem 3 % % %
% % % % % % % % % % %
\begin{exercise}
	
\end{exercise}
\begin{proof}[Solution]
	
\end{proof}

% % % % % % % % % % %
% % % Problem 4 % % %
% % % % % % % % % % %
\begin{exercise}
	
\end{exercise}
\begin{proof}[Solution]
	
\end{proof}

% % % % % % % % % % %
% % % Problem 5 % % %
% % % % % % % % % % %
\begin{exercise}
	
\end{exercise}
\begin{proof}[Solution]
	
\end{proof}

% % % % % % % % % % %
% % % Problem 6 % % %
% % % % % % % % % % %
\begin{exercise}
	
\end{exercise}
\begin{proof}[Solution]
	
\end{proof}


\end{document}