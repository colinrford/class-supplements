\documentclass[12pt,a4]{article}

\usepackage{amsfonts,amsmath,amssymb,amsthm}
\usepackage{amscd}
\usepackage{mathtools}
\usepackage{geometry, algorithmicx} 
\usepackage[noend]{algpseudocode}
\usepackage{subfig}
\usepackage{graphicx}
\usepackage{dsfont}
\usepackage{fancyhdr}
\usepackage{enumerate}
\usepackage{hyperref}
\hypersetup{
	colorlinks,
	citecolor=black,
	filecolor=black,
	linkcolor=black,
	urlcolor=black
}
\usepackage{pgf, tikz}
\usetikzlibrary{shapes,snakes}
\usetikzlibrary{arrows, automata}
\theoremstyle{definition}


\fancyhf{}
\fancyhead[LE,RO]{\thepage}
\fancyhead[CE]{\Author}
\fancyhead[CO]{\Title}
\renewcommand\headrulewidth{0pt}
\pagestyle{fancy}

\author{Colin Ford}
\title{Ascher - Chapter 14 Exercises}
\date{}

\makeatletter
\let\Title\@title
\makeatother

\newtheorem*{theorem*}{Theorem}
\newtheorem*{proposition*}{Proposition}
\newtheorem{problem}{Problem}
\newtheorem*{problem*}{Problem}
\newtheorem{exercise}{Exercise}
\newtheorem{lemma}{Lemma}
\newtheorem*{definition*}{Definition}
\newtheorem*{lemma*}{Lemma}
\newtheorem*{claim*}{Claim}
\newtheorem*{example}{Example}

\setcounter{exercise}{-1}

\begin{document}

\maketitle

% % % % % % % % % % %
% % % Exercise 0 % % %
% % % % % % % % % % %
\begin{exercise}[Review Questions]
	\begin{enumerate}[(a)]
		% % a % %
		\item How does numerical differentiation differ from manual or symbolic differentiation?
		
		% % b % %
		\item Define order of accuracy.
		
		% % c % %
		\item 
	
		% % d % %
		\item 
		
		% % e % %
		\item 
		
		% % f % %
		\item 
		
		% % g % %
		\item 
			
	\end{enumerate}
\end{exercise}
\begin{proof}[Solution]
	\begin{enumerate}[(a)]
		% % a % %
		\item 
		
		% % b % %
		\item The order of accuracy refers to the order $n$ of the truncation error $O(h^n)$ associated with a derivative approximation. 
		
		% % c % %
		\item 
		
		% % d % %
		\item 
		
		% % e % %
		\item 
		
		% % f % %
		\item 
		
		% % g % %
		\item 
		
	\end{enumerate}
\end{proof}

% % % % % % % % % % %
% % % Exercise 1 % % %
% % % % % % % % % % %
\begin{exercise}
	Derive a difference formula for the fourth derivative of $f$ at $x_0$ using Taylor's expansions at $x_0 \pm h$ and $x_0 \pm 2 h$. How many points will be used in total and what is the expected order of the resulting formula?
\end{exercise}
\begin{proof}[Solution]
	From the remarks at the end of page 412, we expect to use $5$ mesh points to approximate the fourth derivative and we expect the order to be equal to $5$. 
\end{proof}

% % % % % % % % % % %
% % % Exercise 2 % % %
% % % % % % % % % % %
\begin{exercise}
	
\end{exercise}
\begin{proof}[Solution]
	
\end{proof}

% % % % % % % % % % %
% % % Exercise 3 % % %
% % % % % % % % % % %
\begin{exercise}
	
\end{exercise}
\begin{proof}[Solution]
	
\end{proof}

% % % % % % % % % % %
% % % Exercise 4 % % %
% % % % % % % % % % %
\begin{exercise}
	
\end{exercise}
\begin{proof}[Solution]
	
\end{proof}

% % % % % % % % % % %
% % % Exercise 5 % % %
% % % % % % % % % % %
\begin{exercise}
	
\end{exercise}
\begin{proof}[Solution]
	
\end{proof}

% % % % % % % % % % %
% % % Exercise 6 % % %
% % % % % % % % % % %
\begin{exercise}
	
\end{exercise}
\begin{proof}[Solution]
	
\end{proof}

% % % % % % % % % % %
% % % Exercise 7 % % %
% % % % % % % % % % %
\begin{exercise}
	
\end{exercise}
\begin{proof}[Solution]
	
\end{proof}

% % % % % % % % % % %
% % % Exercise 8 % % %
% % % % % % % % % % %
\begin{exercise}
	
\end{exercise}
\begin{proof}[Solution]
	
\end{proof}

% % % % % % % % % % %
% % % Exercise 9 % % %
% % % % % % % % % % %
\begin{exercise}
	
\end{exercise}
\begin{proof}[Solution]
	
\end{proof}

% % % % % % % % % % %
% % % Exercise 10 % % %
% % % % % % % % % % %
\begin{exercise}
	(later)
\end{exercise}

% % % % % % % % % % %
% % % Exercise 11 % % %
% % % % % % % % % % %
\begin{exercise}
	(later)
\end{exercise}

% % % % % % % % % % %
% % % Exercise 12 % % %
% % % % % % % % % % %
\begin{exercise}
	(later)
\end{exercise}

% % % % % % % % % % %
% % % Exercise 13 % % %
% % % % % % % % % % %
\begin{exercise}
	
\end{exercise}
\begin{proof}[Solution]
	
\end{proof}

% % % % % % % % % % %
% % % Exercise 14 % % %
% % % % % % % % % % %
\begin{exercise}
	
\end{exercise}
\begin{proof}[Solution]
	
\end{proof}

% % % % % % % % % % %
% % % Exercise 15 % % %
% % % % % % % % % % %
\begin{exercise}
	
\end{exercise}
\begin{proof}[Solution]
	
\end{proof}

% % % % % % % % % % %
% % % Exercise 16 % % %
% % % % % % % % % % %
\begin{exercise}
	(later) 
\end{exercise}

% % % % % % % % % % %
% % % Exercise 17 % % %
% % % % % % % % % % %
\begin{exercise}
	(later) 
\end{exercise}
\begin{proof}[Solution]
	
\end{proof}

\end{document}