\documentclass[12pt,a4]{article}

\usepackage{amsfonts,amsmath,amssymb,amsthm}
\usepackage{amscd}
\usepackage{mathtools}
\usepackage{geometry, algorithmicx} 
\usepackage[noend]{algpseudocode}
\usepackage{subfig}
\usepackage{graphicx}
\usepackage{dsfont}
\usepackage{fancyhdr}
\usepackage{enumerate}
\usepackage{hyperref}
\hypersetup{
	colorlinks,
	citecolor=black,
	filecolor=black,
	linkcolor=black,
	urlcolor=black
}
\usepackage{pgf, tikz}
\usetikzlibrary{shapes,snakes}
\usetikzlibrary{arrows, automata}
\theoremstyle{definition}


\fancyhf{}
\fancyhead[LE,RO]{\thepage}
\fancyhead[CE]{\Author}
\fancyhead[CO]{\Title}
\renewcommand\headrulewidth{0pt}
\pagestyle{fancy}

\author{Colin Ford}
\title{Ascher - Chapter 14 Exercises}
\date{}

\makeatletter
\let\Title\@title
\makeatother

\newtheorem*{theorem*}{Theorem}
\newtheorem*{proposition*}{Proposition}
\newtheorem{problem}{Problem}
\newtheorem*{problem*}{Problem}
\newtheorem{exercise}{Exercise}
\newtheorem{lemma}{Lemma}
\newtheorem*{definition*}{Definition}
\newtheorem*{lemma*}{Lemma}
\newtheorem*{claim*}{Claim}
\newtheorem*{example}{Example}

\setcounter{exercise}{-1}

\begin{document}

\maketitle

% % % % % % % % % % %
% % % Exercise 0 % % %
% % % % % % % % % % %
\begin{exercise}[Review Questions]
	\begin{enumerate}[(a)]
		% % a % %
		\item How does numerical differentiation differ from manual or symbolic differentiation?
		
		% % b % %
		\item Define order of accuracy.
		
		% % c % %
		\item 
	
		% % d % %
		\item 
		
		% % e % %
		\item 
		
		% % f % %
		\item 
		
		% % g % %
		\item 
			
	\end{enumerate}
\end{exercise}
\begin{proof}[Solution]
	\begin{enumerate}[(a)]
		% % a % %
		\item 
		
		% % b % %
		\item The order of accuracy refers to the order $n$ of the truncation error $O(h^n)$ associated with a derivative approximation. 
		
		% % c % %
		\item 
		
		% % d % %
		\item 
		
		% % e % %
		\item 
		
		% % f % %
		\item 
		
		% % g % %
		\item 
		
	\end{enumerate}
\end{proof}

% % % % % % % % % % %
% % % Exercise 1 % % %
% % % % % % % % % % %
\begin{exercise}
	Derive a difference formula for the fourth derivative of $f$ at $x_0$ using Taylor's expansions at $x_0 \pm h$ and $x_0 \pm 2 h$. How many points will be used in total and what is the expected order of the resulting formula?
\end{exercise}
\begin{proof}[Solution]
	From the remarks at the end of page 412, we expect to use $5$ mesh points to approximate the fourth derivative and we expect the order to be equal to $2$. 
	
	First we write
	
	\begin{align*}
	f(x_0 \pm h) &= f(x_0) \pm h f'(x_0) + \frac{h^2}{2} f''(x_0) \pm \frac{h^3}{6} f'''(x_0) + \frac{h^4}{24} f^{(4)}(x_0) \\
	 &\quad \pm \frac{h^5}{120} f^{(5)}(x_0) + \frac{h^6}{720} f^{(6)}(\xi_{p_i}) {,} \\
	f(x_0 \pm 2 h) &= f(x_0) \pm 2 h f'(x_0) + \frac{2^2 h^2}{2} f''(x_0) \pm \frac{2^3 h^3}{6} f'''(x_0) + \frac{2^4 h^4}{24} f^{(4)}(x_0) \\
	&\quad \pm \frac{2^5 h^5}{120} f^{(5)}(x_0) + \frac{2^6 h^6}{720} f^{(6)}(\xi_{q_i}) {,}
	\end{align*}
	
	\noindent where $p_1 = 1$ and $p_2 = 2$ for $f(x_0 + h)$ and $f(x_0 - h)$, respectively; and $q_1 = 3$ and $q_2 = 4$ for $f(x_0 + 2 h)$ and $f(x_0 - 2 h)$, respectively. Adding the pair $f(x_0 \pm h)$ from each other and likewise for the pair $f(x_0 \pm 2 h)$ leads to two centered sixth order approximations for $f^{(4)}(x_0)$:
	
	\begin{align*}
	f(x_0 + h) + f(x_0 - h) &= 2 f + 2 \frac{h^2}{2} f''(x_0) + 2 \frac{h^4}{24} f^{(4)} (x_0) \\
	 &\quad + \frac{h^6}{720} \left( f^{(6)}(\xi_1) + f^{(6)}(\xi_2) \right) {,} \tag{1.1} \label{eq:1.1} \\ 
	f(x_0 + 2 h) + f(x_0 - 2 h) &= 2 f + 2 \frac{h^2}{2} f''(x_0) + 2 \frac{h^4}{24} f^{(4)} (x_0) \\
	 &\quad + 2 \frac{h^6}{720} \left( f^{(6)}(\xi_3) + f^{(6)}(\xi_4) \right) {,} \tag{1.2} \label{eq:1.2} 
	\end{align*}
	
	\noindent where we can find a single $\xi_1'$ for \hyperref[eq:1.1]{\eqref{eq:1.1}} and a single $\xi_2'$ for \hyperref[eq:1.2]{\eqref{eq:1.2}}. Then, from these, we can obtain an approximation for $f^{(4)}(x_0)$. Indeed, with some rearranging, we have
	
	\begin{align*}
	\hyperref[eq:1.2]{\eqref{eq:1.2}} - 4 \cdot \hyperref[eq:1.1]{\eqref{eq:1.1}} &= f(x_0 + 2 h) + f(x_0 - 2 h) - 2 f(x_0)  \\ 
	 &\quad - 4 f(x_0 + h) - 4 f(x_0 - h) - 8 f(x_0) \\
	 &= h^4 f^{(4)}(x_0) + \frac{8 h^6}{720} \left( 16 f^{(6)}(\xi_2') - f^{(6)}(\xi_1') \right) {.}
	\end{align*}
	
	\noindent Solving for $f^{(4)}(x_0)$ gives a formula for the fourth derivative, written as 
	
	\begin{align*}
	f^{(4)}(x_0) &= \frac{f(x_0 + 2 h) + f(x_0 - 2 h) - 2 f(x_0) - 4 f(x_0 + h) - 4 f(x_0 - h) - 8 f(x_0)}{h^4} \\
	 &\quad - \frac{8 h^2}{720} \left( 16 f^{(6)}(\xi_2') - f^{(6)}(\xi_1') \right) {.} 
	\end{align*}
\end{proof}

% % % % % % % % % % %
% % % Exercise 2 % % %
% % % % % % % % % % %
\begin{exercise}
	
\end{exercise}
\begin{proof}[Solution]
	
\end{proof}

% % % % % % % % % % %
% % % Exercise 3 % % %
% % % % % % % % % % %
\begin{exercise}
	
\end{exercise}
\begin{proof}[Solution]
	
\end{proof}

% % % % % % % % % % %
% % % Exercise 4 % % %
% % % % % % % % % % %
\begin{exercise}
	
\end{exercise}
\begin{proof}[Solution]
	
\end{proof}

% % % % % % % % % % %
% % % Exercise 5 % % %
% % % % % % % % % % %
\begin{exercise}
	
\end{exercise}
\begin{proof}[Solution]
	
\end{proof}

% % % % % % % % % % %
% % % Exercise 6 % % %
% % % % % % % % % % %
\begin{exercise}
	
\end{exercise}
\begin{proof}[Solution]
	
\end{proof}

% % % % % % % % % % %
% % % Exercise 7 % % %
% % % % % % % % % % %
\begin{exercise}
	
\end{exercise}
\begin{proof}[Solution]
	
\end{proof}

% % % % % % % % % % %
% % % Exercise 8 % % %
% % % % % % % % % % %
\begin{exercise}
	
\end{exercise}
\begin{proof}[Solution]
	
\end{proof}

% % % % % % % % % % %
% % % Exercise 9 % % %
% % % % % % % % % % %
\begin{exercise}
	
\end{exercise}
\begin{proof}[Solution]
	
\end{proof}

% % % % % % % % % % %
% % % Exercise 10 % % %
% % % % % % % % % % %
\begin{exercise}
	(later)
\end{exercise}

% % % % % % % % % % %
% % % Exercise 11 % % %
% % % % % % % % % % %
\begin{exercise}
	(later)
\end{exercise}

% % % % % % % % % % %
% % % Exercise 12 % % %
% % % % % % % % % % %
\begin{exercise}
	(later)
\end{exercise}

% % % % % % % % % % %
% % % Exercise 13 % % %
% % % % % % % % % % %
\begin{exercise}
	
\end{exercise}
\begin{proof}[Solution]
	
\end{proof}

% % % % % % % % % % %
% % % Exercise 14 % % %
% % % % % % % % % % %
\begin{exercise}
	
\end{exercise}
\begin{proof}[Solution]
	
\end{proof}

% % % % % % % % % % %
% % % Exercise 15 % % %
% % % % % % % % % % %
\begin{exercise}
	
\end{exercise}
\begin{proof}[Solution]
	
\end{proof}

% % % % % % % % % % %
% % % Exercise 16 % % %
% % % % % % % % % % %
\begin{exercise}
	(later) 
\end{exercise}

% % % % % % % % % % %
% % % Exercise 17 % % %
% % % % % % % % % % %
\begin{exercise}
	(later) 
\end{exercise}
\begin{proof}[Solution]
	
\end{proof}

\end{document}