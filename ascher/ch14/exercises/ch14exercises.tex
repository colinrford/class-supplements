\documentclass[12pt,a4]{article}

\usepackage{amsfonts,amsmath,amssymb,amsthm}
\usepackage{amscd}
\usepackage{mathtools}
\usepackage{geometry, algorithmicx} 
\usepackage[noend]{algpseudocode}
\usepackage{subfig}
\usepackage{graphicx}
\usepackage{dsfont}
\usepackage{fancyhdr}
\usepackage{enumerate}
\usepackage{hyperref}
\hypersetup{
	colorlinks,
	citecolor=black,
	filecolor=black,
	linkcolor=black,
	urlcolor=black
}
\usepackage{pgf, tikz}
\usetikzlibrary{shapes,snakes}
\usetikzlibrary{arrows, automata}
\theoremstyle{definition}


\fancyhf{}
\fancyhead[LE,RO]{\thepage}
\fancyhead[CE]{\Author}
\fancyhead[CO]{\Title}
\renewcommand\headrulewidth{0pt}
\pagestyle{fancy}

\author{Colin Ford}
\title{Ascher - Chapter 14 Exercises}
\date{}

\makeatletter
\let\Title\@title
\makeatother

\newtheorem*{theorem*}{Theorem}
\newtheorem*{proposition*}{Proposition}
\newtheorem{problem}{Problem}
\newtheorem*{problem*}{Problem}
\newtheorem{exercise}{Exercise}
\newtheorem{lemma}{Lemma}
\newtheorem*{definition*}{Definition}
\newtheorem*{lemma*}{Lemma}
\newtheorem*{claim*}{Claim}
\newtheorem*{example}{Example}

\setcounter{exercise}{-1}

\begin{document}

\maketitle

% % % % % % % % % % %
% % % Exercise 0 % % %
% % % % % % % % % % %
\begin{exercise}[Review Questions]
	\begin{enumerate}[(a)]
		% % a % %
		\item What is a piecewise polynomial?
		
		% % b % %
		\item State at least three shortcomings of polynomial interpolation that are improved upon by piecewise polynomial interpolation.
		
		% % c % %
		\item State at least one advantage and one disadvantage of piecewise constant interpolation.
	
		% % d % %
		\item State at least two advantages and two disadvantages of broken line interpolation.
		
		% % e % %
		\item What is a piecewise cubic Hermite?
		
		% % f % %
		\item In what ways are the cubic spline and piecewise cubic Hermite useful interpolants?
		
		% % g % %
		\item Define the different end conditions for cubic spline interpolation, giving rise to the natural, complete, and not-a-knot variants. Which of these is most suitable for general purpose implementation, and why?
		
		% % h % %
		\item 
		
		% % i % %
		\item 
		
		% % j % %
		\item 
		
		% % k % %
		\item 
		
		% % l % %
		\item 
		
		% % m % %
		\item 
		
		% % n % %
		\item 
		
		% % o % %
		\item 
		
		% % p % %
		\item 
		
		% % q % %
		\item 
		
		% % r % %
		\item 
		
		% % s % %
		\item 
		
		% % t % %
		\item 
		
	\end{enumerate}
\end{exercise}
\begin{proof}[Solution]
	\begin{enumerate}[(a)]
		% % a % %
		\item A \emph{piecewise polynomial} is a function that is a polynomial on each of its subdomains, but which may differ from subdomain to subdomain. 
		
		% % b % %
		\item An interpolant of the form discussed in the previous chapter is not always suitable because 
		
		\begin{itemize}
			\item The error term 
			
			\[
			f(x) - p_n(x) = \frac{f^{(n + 1)} (\xi)}{(n + 1)!} \prod_{i = 0}^n (x - x_i)
			\]
			
			may not be small if $\frac{\| f^{(n + 1)} \|}{(n + 1)!}$ is not. A previous example illustrates this.
			
			\item High order polynomials tend to oscillate ``unreasonably.''
			
			\item Data often are only piecewise smooth, whereas polynomials are infinitely differentiable. The high derivatives $f^{(n + 1)}$ may blow up in such a case, which again yields a large error term. 
			
			\item No locality: changing one data value may drastically alter the entire interpolant. 
		\end{itemize}
		
		% % c % %
		\item This method is convenient when we know nothing of continuity or smoothness of the function $f$. Piecewise constant interpolation is often not smooth enough for applications. 
		
		% % d % %
		\item Among the advantages of broken line interpolation are its simplicity as well as that the maximum and minimum values are at the data points, i.e., no new extremum point is added during the interpolation process. The piecewise linear interpolant is not continuously differentiable everywhere, and as a result, it is often not smooth enough. 
		
		% % e % %
		\item 
		
		% % f % %
		\item 
		
		% % g % %
		\item 
		
		% % h % %
		\item 
		
		% % i % %
		\item 
		
		% % j % %
		\item 
		
		% % k % %
		\item 
		
		% % l % %
		\item 
		
		% % m % %
		\item 
		
		% % n % %
		\item 
		
		% % o % %
		\item 
		
		% % p % %
		\item 
		
		% % q % %
		\item 
		
		% % r % %
		\item 
		
		% % s % %
		\item 
		
		% % t % %
		\item 
	\end{enumerate}
\end{proof}

% % % % % % % % % % %
% % % Exercise 1 % % %
% % % % % % % % % % %
\begin{exercise}
	
\end{exercise}
\begin{proof}[Solution]
	
\end{proof}

% % % % % % % % % % %
% % % Exercise 2 % % %
% % % % % % % % % % %
\begin{exercise}
	
\end{exercise}
\begin{proof}[Solution]
	
\end{proof}

% % % % % % % % % % %
% % % Exercise 3 % % %
% % % % % % % % % % %
\begin{exercise}
	
\end{exercise}
\begin{proof}[Solution]
	
\end{proof}

% % % % % % % % % % %
% % % Exercise 4 % % %
% % % % % % % % % % %
\begin{exercise}
	
\end{exercise}
\begin{proof}[Solution]
	
\end{proof}

% % % % % % % % % % %
% % % Exercise 5 % % %
% % % % % % % % % % %
\begin{exercise}
	
\end{exercise}
\begin{proof}[Solution]
	
\end{proof}

% % % % % % % % % % %
% % % Exercise 6 % % %
% % % % % % % % % % %
\begin{exercise}
	
\end{exercise}
\begin{proof}[Solution]
	
\end{proof}

% % % % % % % % % % %
% % % Exercise 7 % % %
% % % % % % % % % % %
\begin{exercise}
	
\end{exercise}
\begin{proof}[Solution]
	
\end{proof}

% % % % % % % % % % %
% % % Exercise 8 % % %
% % % % % % % % % % %
\begin{exercise}
	
\end{exercise}
\begin{proof}[Solution]
	
\end{proof}

% % % % % % % % % % %
% % % Exercise 9 % % %
% % % % % % % % % % %
\begin{exercise}
	
\end{exercise}
\begin{proof}[Solution]
	
\end{proof}

% % % % % % % % % % %
% % % Exercise 10 % % %
% % % % % % % % % % %
\begin{exercise}
	(later)
\end{exercise}

% % % % % % % % % % %
% % % Exercise 11 % % %
% % % % % % % % % % %
\begin{exercise}
	(later)
\end{exercise}

% % % % % % % % % % %
% % % Exercise 12 % % %
% % % % % % % % % % %
\begin{exercise}
	(later)
\end{exercise}

% % % % % % % % % % %
% % % Exercise 13 % % %
% % % % % % % % % % %
\begin{exercise}
	
\end{exercise}
\begin{proof}[Solution]
	
\end{proof}

% % % % % % % % % % %
% % % Exercise 14 % % %
% % % % % % % % % % %
\begin{exercise}
	
\end{exercise}
\begin{proof}[Solution]
	
\end{proof}

% % % % % % % % % % %
% % % Exercise 15 % % %
% % % % % % % % % % %
\begin{exercise}
	
\end{exercise}
\begin{proof}[Solution]
	
\end{proof}

% % % % % % % % % % %
% % % Exercise 16 % % %
% % % % % % % % % % %
\begin{exercise}
	(later) 
\end{exercise}

% % % % % % % % % % %
% % % Exercise 17 % % %
% % % % % % % % % % %
\begin{exercise}
	(later) 
\end{exercise}

% % % % % % % % % % %
% % % Exercise 18 % % %
% % % % % % % % % % %
\begin{exercise}
	(later) 
\end{exercise}

% % % % % % % % % % %
% % % Exercise 19 % % %
% % % % % % % % % % %
\begin{exercise}
	(later) 
\end{exercise}

% % % % % % % % % % %
% % % Exercise 20 % % %
% % % % % % % % % % %
\begin{exercise}
	(later) 
\end{exercise}

% % % % % % % % % % %
% % % Exercise 21 % % %
% % % % % % % % % % %
\begin{exercise}
	(later) 
\end{exercise}

% % % % % % % % % % %
% % % Exercise 22 % % %
% % % % % % % % % % %
\begin{exercise}
	
\end{exercise}
\begin{proof}[Solution]
	
\end{proof}

% % % % % % % % % % %
% % % Exercise 23 % % %
% % % % % % % % % % %
\begin{exercise}
	(later) 
\end{exercise}
\begin{proof}[Solution]
	
\end{proof}

\end{document}