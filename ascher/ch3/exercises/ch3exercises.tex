\documentclass[12pt,a4]{article}

\usepackage{amsfonts,amsmath,amssymb,amsthm}
\usepackage{amscd}
\usepackage{mathtools}
\usepackage{geometry, algorithmicx} 
\usepackage[noend]{algpseudocode}
\usepackage{subfig}
\usepackage{graphicx}
\usepackage{dsfont}
\usepackage{fancyhdr}
\usepackage{enumerate}
\usepackage{pgf, tikz}
\usetikzlibrary{shapes,snakes}
\usetikzlibrary{arrows, automata}
\theoremstyle{definition}


\fancyhf{}
\fancyhead[LE,RO]{\thepage}
\fancyhead[CE]{\Author}
\fancyhead[CO]{\Title}
\renewcommand\headrulewidth{0pt}
\pagestyle{fancy}

\author{Colin Ford}
\title{Ascher - Chapter 3 Exercises}
\date{}

\makeatletter
\let\Title\@title
\makeatother

\newtheorem*{theorem*}{Theorem}
\newtheorem*{proposition*}{Proposition}
\newtheorem{problem}{Problem}
\newtheorem*{problem*}{Problem}
\newtheorem{exercise}{Exercise}
\newtheorem{lemma}{Lemma}
\newtheorem*{definition*}{Definition}
\newtheorem*{lemma*}{Lemma}
\newtheorem*{claim*}{Claim}
\newtheorem*{example}{Example}

\begin{document}

\maketitle

% % % % % % % % % % %
% % % Problem 1 % % %
% % % % % % % % % % %
\begin{exercise}[Review Questions]
	\begin{enumerate}[(a)]
		% % a % %
		\item What is a nonlinear equation?
		
		% % b % %
		\item Is the bisection method (i) efficient? (ii) robust? Does it (iii) require a minimum amount of additional knowledge? (iv) require $f$ to satisfy only minimum smoothness properties? (v) generalize easily to several functions in several variables?
		
		% % c % %
		\item Answer similar questions for the Newton and Secant methods.
		
		% % d % %
		\item State at least one advantage and one disadvantage of the recursive implementation of the bisection method over the iterative nonrecursive implementation.
		
		% % e % %
		\item In what way is the fixed point iteration a \emph{family} of methods, rather than just one method like bisection or secant?
		
		% % f % %
		\item What is the basic condition for convergence of the fixed point iteration, and how does the speed of convergence relate to the derivative of the iteration function $g$?
		
		% % g % %
		\item Suppose a given fixed point iteration does not converge: does this mean that there is no root in the relevant interval? Answer a similar question for the Newton and secant methods. 
		
		% % h % %
		\item State at least two advantages and two disadvantages of Newton's method. 

		% % i % %
		\item What are order of convergence and rate of convergence, and how do they relate?
		
		% % j % %
		\item State at least one advantage and one disadvantage of the secant method over Newton's.
		
		% % k % %
		\item In what situation does Newton's method converge only linearly?
		
		% % l % %
		\item Explain the role that roundoff errors  play in the convergence of solvers for nonlinear equations, and explain their relationship with convergence errors.
		
		% % m % %
		\item State a similarity and a difference between the  problem of minimizing a function $\varphi(x)$ and that of solving the nonlinear equation $\varphi'(x) = 0$. 
		
		% % n % %
		\item State what a convex function is, and explain what happens if an objective function is convex. 
	\end{enumerate}
\end{exercise}
\begin{proof}[Solution]
	\begin{enumerate}[(a)]
		% % a % %
		\item A function $f$ is \emph{linear} if it satisfies the following two properties: 
		
		\begin{align*}
		f(x + y) &= f(x) + f(y) \\
		f(\alpha x) &= \alpha f(x) 
		\end{align*}
		
		where the first property is called additivity and the second homogeneity. Any function which does not satisfy these properties is called \emph{nonlinear}. 
		
		% % b % %
		\item 
		\begin{enumerate}[(i)]
			% % i % %
			\item 
			
			% % ii % %
			\item 
			
			% % iii % %
			\item 
		\end{enumerate}
		
		% % c % %
		\item 
		
		% % d % %
		\item 
		
		% % e % %
		\item 
		
		% % f % %
		\item 
		
		% % g % %
		\item 
		
		% % h % %
		\item 
		
		% % i % %
		\item 
		
		% % j % %
		\item 
		
		% % k % %
		\item 
		
		% % l % %
		\item 
		
		% % m % %
		\item 
		
		% % n % %
		\item 
	\end{enumerate}
\end{proof}

% % % % % % % % % % %
% % % Problem 2 % % %
% % % % % % % % % % %
\begin{exercise}
	 
\end{exercise}
\begin{proof}[Solution]
	
\end{proof}

% % % % % % % % % % %
% % % Problem 3 % % %
% % % % % % % % % % %
\begin{exercise}
	
\end{exercise}
\begin{proof}[Solution]
	
\end{proof}

% % % % % % % % % % %
% % % Problem 4 % % %
% % % % % % % % % % %
\begin{exercise}
	
\end{exercise}
\begin{proof}[Solution]
	
\end{proof}

% % % % % % % % % % %
% % % Problem 5 % % %
% % % % % % % % % % %
\begin{exercise}
	
\end{exercise}
\begin{proof}[Solution]
	
\end{proof}

% % % % % % % % % % %
% % % Problem 6 % % %
% % % % % % % % % % %
\begin{exercise}
	
\end{exercise}
\begin{proof}[Solution]
	
\end{proof}


\end{document}