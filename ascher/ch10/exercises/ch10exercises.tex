\documentclass[12pt,a4]{article}

\usepackage{amsfonts,amsmath,amssymb,amsthm}
\usepackage{amscd}
\usepackage{mathtools}
\usepackage{geometry, algorithmicx} 
\usepackage[noend]{algpseudocode}
\usepackage{subfig}
\usepackage{graphicx}
\usepackage{dsfont}
\usepackage{fancyhdr}
\usepackage{enumerate}
\usepackage{pgf, tikz}
\usetikzlibrary{shapes,snakes}
\usetikzlibrary{arrows, automata}
\theoremstyle{definition}


\fancyhf{}
\fancyhead[LE,RO]{\thepage}
\fancyhead[CE]{\Author}
\fancyhead[CO]{\Title}
\renewcommand\headrulewidth{0pt}
\pagestyle{fancy}

\author{Colin Ford}
\title{Ascher - Chapter 10 Exercises}
\date{}

\makeatletter
\let\Title\@title
\makeatother

\newtheorem*{theorem*}{Theorem}
\newtheorem*{proposition*}{Proposition}
\newtheorem{problem}{Problem}
\newtheorem*{problem*}{Problem}
\newtheorem{exercise}{Exercise}
\newtheorem{lemma}{Lemma}
\newtheorem*{definition*}{Definition}
\newtheorem*{lemma*}{Lemma}
\newtheorem*{claim*}{Claim}
\newtheorem*{example}{Example}

\begin{document}

\maketitle

% % % % % % % % % % %
% % % Problem 1 % % %
% % % % % % % % % % %
\begin{exercise}[Review Questions]
	\begin{enumerate}[(a)]
		% % a % %
		\item Distinguish between the terms data fitting, interpolation, and polynomial interpolation.
		
		% % b % %
		\item Distinguish between (discrete) data fitting and approximating a given function.
		
		% % c % %
		\item What are basis functions? Does an approximant $v(x)$ that is written as a linear combination of basis functions have to be linear in $x$?
		
		% % d % %
		\item An interpolating polynomial is unique regardless of the choice of the basis. Explain why.
		
		% % e % %
		\item State one advantage and two disadvantages of using the monomial basis for polynomial interpolation.
		
		% % f % %
		\item What are Lagrange polynomials? How are they used for polynomial interpolation?
		
		% % g % %
		\item What are barycentric weights?
		
		% % h % %
		\item State the main advantages and the main disadvantage for using the Lagrange representation.

		% % i % %
		\item What is a divided difference table and how is it constructed?
		
		% % j % %
		\item Write down the formula for polynomial interpolation in Newton form.
		
		% % k % %
		\item State two advantages and two disadvantages for using the Newton representation for polynomial interpolation.
		
		% % l % %
		\item Describe the linear systems that are solved for the monomial basis, the Lagrange representation, and the Newton representation.
		
		% % m % %
		\item Describe the connection between the $k$th divided difference of a function $f$ and its $k$th derivative.
		
		% % n % %
		\item Provide an expression for the error in polynomial interpolation as well as an error bound expression.
		
		% % o % %
		\item How does the smoothness of a function and its derivatives affect the quality of polynomial interpolants that approximate it, in general?
		
		% % p % %
		\item Give an example where the error bound is attained. 
		
		% % q % %
		\item When we interpolate a function $f$ given only data points, i.e., we do not know $f$ or its derivatives, how can we gauge the accuracy of our approximation?
		
		% % r % %
		\item What are Chebyshev points and why are they important?
		
		% % s % %
		\item Describe osculating interpolation. How is it different from the usual polynomial interpolation?
		
		% % t % %
		\item What is a Hermite cubic interpolant?
		
	\end{enumerate}
\end{exercise}
\begin{proof}[Solution]
	\begin{enumerate}[(a)]
		% % a % %
		\item 
		
		% % b % %
		\item 
		
		% % c % %
		\item 
		
		% % d % %
		\item 
		
		% % e % %
		\item 
		
		% % f % %
		\item 
		
		% % g % %
		\item 
		
		% % h % %
		\item 
		
		% % i % %
		\item 
		
		% % j % %
		\item 
		
		% % k % %
		\item 
		
		% % l % %
		\item 
		
		% % m % %
		\item 
		
		% % n % %
		\item 
	\end{enumerate}
\end{proof}

% % % % % % % % % % %
% % % Problem 2 % % %
% % % % % % % % % % %
\begin{exercise}
	 
\end{exercise}
\begin{proof}[Solution]
	
\end{proof}

% % % % % % % % % % %
% % % Problem 3 % % %
% % % % % % % % % % %
\begin{exercise}
	
\end{exercise}
\begin{proof}[Solution]
	
\end{proof}

% % % % % % % % % % %
% % % Problem 4 % % %
% % % % % % % % % % %
\begin{exercise}
	
\end{exercise}
\begin{proof}[Solution]
	
\end{proof}

% % % % % % % % % % %
% % % Problem 5 % % %
% % % % % % % % % % %
\begin{exercise}
	
\end{exercise}
\begin{proof}[Solution]
	
\end{proof}

% % % % % % % % % % %
% % % Problem 6 % % %
% % % % % % % % % % %
\begin{exercise}
	
\end{exercise}
\begin{proof}[Solution]
	
\end{proof}


\end{document}