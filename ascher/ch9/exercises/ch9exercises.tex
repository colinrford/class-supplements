\documentclass[12pt,a4]{article}

\usepackage{amsfonts,amsmath,amssymb,amsthm}
\usepackage{amscd}
\usepackage{mathtools}
\usepackage{geometry, algorithmicx} 
\usepackage[noend]{algpseudocode}
\usepackage{subfig}
\usepackage{graphicx}
\usepackage{dsfont}
\usepackage{fancyhdr}
\usepackage{enumerate}
\usepackage{pgf, tikz}
\usetikzlibrary{shapes,snakes}
\usetikzlibrary{arrows, automata}
\theoremstyle{definition}


\fancyhf{}
\fancyhead[LE,RO]{\thepage}
\fancyhead[CE]{\Author}
\fancyhead[CO]{\Title}
\renewcommand\headrulewidth{0pt}
\pagestyle{fancy}

\author{Colin Ford}
\title{Ascher - Chapter 9 Exercises}
\date{}

\makeatletter
\let\Title\@title
\makeatother

\newtheorem*{theorem*}{Theorem}
\newtheorem*{proposition*}{Proposition}
\newtheorem{problem}{Problem}
\newtheorem*{problem*}{Problem}
\newtheorem{exercise}{Exercise}
\newtheorem{lemma}{Lemma}
\newtheorem*{definition*}{Definition}
\newtheorem*{lemma*}{Lemma}
\newtheorem*{claim*}{Claim}
\newtheorem*{example}{Example}

\begin{document}

\maketitle

% % % % % % % % % % %
% % % Problem 1 % % %
% % % % % % % % % % %
\begin{exercise}[Review Questions]
	\begin{enumerate}[(a)]
		% % a % %
		\item What is constrained optimization and what is unconstrained optimization? State a significant difference between the two.
		
		% % b % %
		\item What is a Jacobian matrix?
		
		% % c % %
		\item How many solutions is a nonlinear system of n algebraic equations expected to have?
		
		% % d % %
		\item What are the fundamental additional difficulties in the numerical solution of problems considered in Section 9.1 as compared to those considered in Chapter 3?
		
		% % e % %
		\item State the condition for quadratic convergence for a nonlinear system and explain its importance.
		
		% % f % %
		\item Explain how minimizing a nonlinear function in several variables (Section 9.2) leads to solving systems of algebraic equations (Section 9.1).
		
		% % g % %
		\item What are the necessary and the sufficient conditions for an unconstrained minimum?
		
		% % h % %
		\item What is the difference between a local minimum and a global minimum? Which of the two types do the methods discussed in this chapter typically attempt to find?

		% % i % %
		\item Why is it important to keep a symmetric positive definite iteration matrix Bk while seeking a minimum of a smooth function $\phi(\mathbf{x})$? Does Newton’s method automatically guarantee this?
		
		% % j % %
		\item Define descent direction and line search, and explain their relationship.
		
		% % k % %
		\item What is a gradient descent method? State two advantages and two disadvantages that it has over Newton’s method for unconstrained optimization.
		
		% % l % %
		\item What is a quasi-Newton method? Name three advantages that such a method may have over Newton’s method.
		
		% % m % %
		\item State the KKT conditions and explain their importance.
		
		% % n % %
		\item What is an active set method? Name a famous active set method for the problem of linear programming.
		
		% % o % %
		\item How does the primal-dual form for linear programming relate to the primal form of such problems?
		
		% % p % %
		\item Define central path and duality gap for linear programming problems.
		
	\end{enumerate}
\end{exercise}
\begin{proof}[Solution]
	\begin{enumerate}[(a)]
		% % a % %
		\item 
		
		% % b % %
		\item 
		
		% % c % %
		\item 
		
		% % d % %
		\item 
		
		% % e % %
		\item 
		
		% % f % %
		\item 
		
		% % g % %
		\item 
		
		% % h % %
		\item 
		
		% % i % %
		\item 
		
		% % j % %
		\item 
		
		% % k % %
		\item 
		
		% % l % %
		\item 
		
		% % m % %
		\item 
		
		% % n % %
		\item 
	\end{enumerate}
\end{proof}

% % % % % % % % % % %
% % % Problem 2 % % %
% % % % % % % % % % %
\begin{exercise}
	 
\end{exercise}
\begin{proof}[Solution]
	
\end{proof}

% % % % % % % % % % %
% % % Problem 3 % % %
% % % % % % % % % % %
\begin{exercise}
	
\end{exercise}
\begin{proof}[Solution]
	
\end{proof}

% % % % % % % % % % %
% % % Problem 4 % % %
% % % % % % % % % % %
\begin{exercise}
	
\end{exercise}
\begin{proof}[Solution]
	
\end{proof}

% % % % % % % % % % %
% % % Problem 5 % % %
% % % % % % % % % % %
\begin{exercise}
	
\end{exercise}
\begin{proof}[Solution]
	
\end{proof}

% % % % % % % % % % %
% % % Problem 6 % % %
% % % % % % % % % % %
\begin{exercise}
	
\end{exercise}
\begin{proof}[Solution]
	
\end{proof}


\end{document}