\documentclass[12pt,a4]{article}

\usepackage{amsfonts,amsmath,amssymb,amsthm}
\usepackage{amscd}
\usepackage{mathtools}
\usepackage{geometry, algorithmicx} 
\usepackage[noend]{algpseudocode}
\usepackage{subfig}
\usepackage{graphicx}
\usepackage{dsfont}
\usepackage{fancyhdr}
\usepackage{enumerate}
\usepackage{hyperref}
\hypersetup{
	colorlinks,
	citecolor=black,
	filecolor=black,
	linkcolor=black,
	urlcolor=black
}
\usepackage{pgf, tikz}
\usetikzlibrary{shapes,snakes}
\usetikzlibrary{arrows, automata}
\theoremstyle{definition}


\fancyhf{}
\fancyhead[LE,RO]{\thepage}
\fancyhead[CE]{\Author}
\fancyhead[CO]{\Title}
\renewcommand\headrulewidth{0pt}
\pagestyle{fancy}

\author{Colin Ford}
\title{Ascher - Chapter 12 Exercises}
\date{}

\makeatletter
\let\Title\@title
\makeatother

\newtheorem*{theorem*}{Theorem}
\newtheorem*{proposition*}{Proposition}
\newtheorem{problem}{Problem}
\newtheorem*{problem*}{Problem}
\newtheorem{exercise}{Exercise}
\newtheorem{lemma}{Lemma}
\newtheorem*{definition*}{Definition}
\newtheorem*{lemma*}{Lemma}
\newtheorem*{claim*}{Claim}
\newtheorem*{example}{Example}

\setcounter{exercise}{-1}

\begin{document}

\maketitle

% % % % % % % % % % %
% % % Exercise 0 % % %
% % % % % % % % % % %
\begin{exercise}[Review Questions]
	\begin{enumerate}[(a)]
		% % a % %
		\item Distinguish between interpolation and best approximation.
		
		% % b % %
		\item What is the difference between continuous and discrete least squares approximation?
		
		% % c % %
		\item Write down the normal equations for least squares approximation.
	
		% % d % %
		\item Write down the Hilbert matrix of size $n \times n$. When does this matrix arise?
		
		% % e % %
		\item What are the conditions for a function $v(x)$ to be the orthogonal projection of a function $f(x)$ onto the space spanned by the two functions $\phi_1(x)$ and $\phi_2(x)$?
		
		% % f % %
		\item What are orthogonal polynomials with respect to a weight function $w(x)$? Why are they useful for numerical computations?
		
		% % g % %
		\item What are the Legendre polynomials, and how can they be used to solve the continuous least squares problem?
		
		% % h % %
		\item Describe how to compute a best approximation on a general interval, $[a, b]$, not necessarily equal to the interval $[-1, 1]$.
		
		% % i % %
		\item Explain what trigonometric polynomials are. Are they really polynomials? Are they orthogonal?
		
		% % j % %
		\item Describe the Gram-Schmidt process and explain its importance.
		
		% % k % %
		\item What are the Chebyshev polynomials? Why do we hear more about them than about other families of orthogonal polynomials?
		
		% % l % %
		\item What is a min-max property?
		
		% % m % %
		\item The weight function $w(x)$ for Chebyshev polynomials satisfies $w(x) \to \infty$ as $x \to \pm 1$. Is this a problem? Explain.
		
	\end{enumerate}
\end{exercise}
\begin{proof}[Solution]
	\begin{enumerate}[(a)]
		% % a % %
		\item The difference from interpolation is that, in best approximation, we no longer require the approximation to pass through the data values. 
		
		% % b % %
		\item On the one hand, the notation becomes simpler in the continuous case since we do not have to work with discrete data points. Additionally, the continuous case relies less heavily on linear algebra, and hence is more closely relatable to interpolation and numerical integration.
		
		% % c % %
		\item The following linear conditions are the \emph{normal equations} for the continuous case:
		
		\begin{align*}
		\tilde{B} \mathbf{c} &= \tilde{\mathbf{b}} {,} \quad \text{where} \\
		\tilde{B}_{j, k} &= \int_a^b \phi_j(x) \phi_k(x) dx {,} \quad \tilde{b}_j = \int_a^b f(x) \phi_j(x) dx {.}
		\end{align*}
		
		% % d % %
		\item With the simplest basis representation on the interval $[0, 1]$, namely, the monomials
		
		\[
		\phi_j(x) = x^j {,} \quad j = 0, 1, \ldots, n {,} 
		\]
		
		\noindent we obtain
		
		\[
		\tilde{B}_{j, k} = \int_0^1 x^{j + k} dx = \frac{1}{j + k + 1} {,} \quad 0 \leq j, k \leq n {.}
		\]
		
		\noindent Thus, $\tilde{B} = \left( \tilde{B}_{j, k} \right)$ turns out the be the notorious \textbf{Hilbert matrix}. 
		
		% % e % %
		\item the residual (approximation error) $r^*(x) = f(x) - v(x)$ of the best least squares approximation $v(x) = \sum_j c_j^* \phi_j(x)$ is orthogonal to the basis functions, satisfying
		
		\[
		\int_a^b r^*(x) \phi_k(x) dx = 0 {,} \quad k = 0, \ldots, n {.}
		\]
		
		\noindent In this case, $n = 2$. Then $v(x)$ is the \textbf{orthogonal projection} of $f(x)$ into the space spanned by the basis functions. 
		
		% % f % %
		\item Often in applications the given data are not equally important everywhere. The weight function helps indicate where data is more or less important. Orthogonal polynomials with respect to a weight function $w(x)$ are still the same polynomials, however, the approximation problem changes to
		
		\[
		\min_{\mathbf{c}} \psi(\mathbf{c}) = \int_a^b w(x) (f(x) - v(x))^2 dx {.}
		\]
		
		\noindent The normal equations also read the same, with the modification in the inner product, so 
		
		\begin{align*}
		\tilde{B} \mathbf{c} &= \tilde{\mathbf{b}} {,} \quad \text{where} \\
		\tilde{B}_{j, k} &= \int_a^b w(x) \phi_j(x) \phi_k(x) dx {,} \quad \tilde{b}_j = \int_a^b w(x) f(x) \phi_j(x) dx {.}
		\end{align*}
		
		% % g % %
		\item The Legendre polynomials form an orthogonal polynomial basis, which can be used as a basis in the continuous least squares problem, and they are defined on the interval $[-1, 1]$ by the three-term recurrence relation 
		
		\begin{align*}
		\phi_0(x) &= 1 {,} \\
		\phi_1(x) &= x {,} \\
		\phi_{j + 1}(x) &= \frac{2 j + 1}{j + 1} x \phi_j(x) - \frac{j}{j + 1} \phi_{j - 1}(x) {,} \quad j \geq 1 {.}
		\end{align*}
		
		% % h % %
		\item We can apply an affine transformation to $x \in [-1, 1]$ which \emph{scales} and \emph{translates} it. That is, we can write a given $t \in [a, b]$ as 
		
		\[
		t = \frac{1}{2} [(b - a) x + (a + b)] \quad \text{or} \quad x = \frac{2 t - a - b}{b - a} {.}
		\]
		
		% % i % %
		\item The \textbf{trigonometric polynomials} are an important non-polynomial orthogonal family of basis functions, given on $[-\pi, \pi]$ by 
		
		\begin{align*}
		\phi_0(x) &= \frac{1}{\sqrt{2 \pi}} {,} \\
		\phi_{2 l - 1}(x) &= \frac{1}{\sqrt{x}} \sin(l x) {,} \\
		\phi_{2 l}(x) &= \frac{1}{\sqrt{\pi}} \cos(l x) 
		\end{align*}
		
		\noindent for $l = 1, \ldots, \frac{n}{2}$ with $n$ assumed even. Here we have 
		
		\[
		\int_{- \pi}^\pi \phi_j(x) \phi_k(x) dx = \delta_{j k } = \begin{cases}
		0 {,} \quad j \neq k {,} \\
		1 {,} \quad j = k {.}
		\end{cases}
		\]
		
		\noindent Thus, $d_j = 1$, and the basis is called \textbf{orthonormal}. 
				
		% % j % %
		\item The Gram-Schmidt process takes a finite, linearly independent set of vectors in a vector space and generates an orthogonal set that spans the same space. 
		
		With the Gram-Schmidt process, for any weight function on any interval, an entire class of orthogonal polynomials is constructed using a simple three-term recurrence relationship. 
		
		% % k % %
		\item For $w(x) = \frac{1}{\sqrt{1 - x^2}}$ and $[a, b] = [-1, 1]$ we obtain the \textbf{Chebyshev polynomials}
		
		\begin{align*}
		\phi_0(x) &= 1 {,} \\
		\phi_1(x) &= x {,} \\
		\phi_2(x) &= 2 x^2 - 1 {,} \\
		\phi_{j + 1}(x) &= 2 x \phi_j(x) - \phi_{j - 1}(x) {.}
		\end{align*}
		
		\noindent Another definition, which is perhaps more elegant, is 
		
		\begin{align*}
		\phi_j(x) = T_j(x) &= \cos(j \ \arccos{x}) \\
		 &= \cos(j \theta) {,} \quad \text{where } x = \cos(\theta) {.}
		\end{align*}
		
		\noindent They arise as a tool in various instances of numerical analysis.
		
		% % l % %
		\item 
		
		% % m % %
		\item The process is still well-defined. One can see this in the verification of orthogonality. Indeed, if we substitute the alternate definition (see (k)), we obtain 
		
		\begin{align*}
		\int_{-1}^1 w(x) T_j(x) T_k(x) dx &= \int_{-1}^1 \frac{T_j(x) T_k(x)}{\sqrt{1 - x^2}} dx \\
		 &= \int_0^\pi \cos(j \theta) \cos(k \theta) d\theta = \begin{cases} 
		 0 {,} \quad j \neq k \\
		 \frac{\pi}{2} {,} \quad j = k > 0 {.}
		 \end{cases}
		\end{align*}
		
		\noindent We see in the transformation from $x$ to $\theta$ that $dx = - \sin(\theta) d\theta$, and $\sqrt{1 - x^2} = \sqrt{1 - \cos^2(\theta)} = \sin(\theta)$, so $\sin(\theta)$ cancels in the integrand. 
	\end{enumerate}
\end{proof}

% % % % % % % % % % %
% % % Exercise 1 % % %
% % % % % % % % % % %
\begin{exercise}
	Using the MATLAB instruction \texttt{cond}, find the condition numbers of Hilbert matrices for $n = 4, 5, \ldots, 12$. Plot these condition numbers as a function of $n$ using \texttt{semilogy}. What are your observations?
\end{exercise}
\begin{proof}[Solution]
	
\end{proof}

% % % % % % % % % % %
% % % Exercise 2 % % %
% % % % % % % % % % %
\begin{exercise}
	Construct the second degree polynomial $q_2(t)$ that approximates $g(t) = \sin(\pi t)$ on the interval $[0, 1]$ by minimizing 
	
	\[
	\int_0^1 [g(t) - q_2(t)]^2 dt {.}
	\]
	
	\noindent Some useful integrals:
	
	\begin{gather}
	\int_0^1 (6 t^2 - 6 t + 1)^2 dt = \frac{1}{5} {,} \quad \int_0^1 \sin(\pi t) dt = \frac{2}{\pi} {,} \\
	\int_0^1 t \sin(\pi t) dt = \frac{1}{\pi} {,} \quad \int_0^1 t^2 \sin(\pi t) dt = \frac{\pi^2 - 4}{\pi^3} {.}
	\end{gather}
\end{exercise}
\begin{proof}[Solution]
	We wish to calculate the quadratic polynomial $q_2(t)$ that minimizes 
	
	\[
	\int_0^1 \left( \sin(\pi t) - q_2(t) \right)^2 dt {.}
	\]
	
	\noindent Here $t = \frac{1}{2}(x + 1)$, or $x = 2 t - 1$. We get
	
	\begin{align*}
	\phi_0(t) &= 1 {,} \\
	\phi_1(t) &= 2 t - 1 {,} \\
	\phi_2(t) &= 6 t^2 - 6 t + 1 {.}
	\end{align*}
	
	\noindent This yields 
	
	\begin{align*}
	\tilde{b}_0 &= \int_0^1 \sin{\pi t} d t = \frac{2}{\pi} {,} \quad d_0 = \| \phi_0 \|^2 = \int_0^1 dt = 1 {,} \\
	\tilde{b}_1 &= \int_0^1 (2 t - 1) \sin(\pi t) dt = 0 {,} \quad d_1 = \| \phi_1 \|^2 = \int_0^1 (2 t - 1)^2 dt = \frac{1}{3} {,} \\
	\tilde{b}_2 &= \int_0^1 (6 t^2 - 6 t + 1) \sin(\pi t) dt = \frac{2(\pi^2 - 12)}{\pi^3} {,} \\
	&\qquad d_2 = \| \phi_2 \|^2 = \int_0^1 (6 t^2 - 6 t + 1)^2 dt = \frac{1}{5} {.}
	\end{align*}
	
	\noindent So, our best approximating quadratic is 
	
	\[
	q_2(t) = \frac{2}{\pi} + \frac{10(\pi^2 - 12)}{\pi^3} (6 t^2 - 6 t + 1) {.}
	\]
\end{proof}

% % % % % % % % % % %
% % % Exercise 3 % % %
% % % % % % % % % % %
\begin{exercise}
	The Legendre polynomials satisfy 
	
	\[
	\int_{-1}^{1} \phi_j(x) \phi_k(x) dx = \begin{cases}
	0 {,} \quad j \neq k {,} \\
	\frac{2}{2 j + 1} {,} \quad j = k {.}
	\end{cases}
	\]
	
	\noindent Suppose that the best fit problem is given on the interval $[a, b]$. 
	
	Show that with the transformation $t = \frac{1}{2} [(b - a) x + (a + b)]$ and a slight change of notation, we have 
	
	\[
	\int_a^b \phi_j(t) \phi_k(t) dt = \begin{cases}
	0 {,} \quad j \neq k {,} \\
	\frac{b - a}{2 j + 1} {,} \quad j = k {.}
	\end{cases}
	\]
\end{exercise}
\begin{proof}[Solution]
	A linear transformation on $x$ will not have an affect on the orthogonality of the basis, so the inner product will still remain $0$ when $j \neq k$. Hence we only need to show the result for when $j = k$. 
\end{proof}

% % % % % % % % % % %
% % % Exercise 4 % % %
% % % % % % % % % % %
\begin{exercise}
	Redo Example 12.1, reconstructing Figure 12.1, using an orthogonal polynomial basis.
\end{exercise}
\begin{proof}[Solution]
	The first three Legendre polynomials are computed for us in the book on Page 371. The fourth we obtain by the three-term recurrence relation
	
	\[
	\phi_4(x) = \frac{1}{8} (35 x^4 - 30 x^2 + 3) {.}
	\]

	\noindent We compute, for each $n = 0, 1, 2, 3, 4$, the best polynomial approximations of degree $\leq n$ of the function $f(x) = \cos(2 \pi x)$ using the Legendre polynomials as basis functions. For this we need the following integrals, many of which are obtained by integration by parts (although one may use the integrals already computed in Example 12.1 to save some time):
	
	\begin{align*}
	\tilde{b}_0 &= \int_{-1}^1 \cos(2 \pi x) dx = 0 {,} \\
	\tilde{b}_1 &= \int_{-1}^1 x \cos(2 \pi x) dx = 0 {,} \\
	\tilde{b}_2 &= \frac{1}{2} \int_{-1}^1 (3 x^2 - 1) \cos(2 \pi x) dx = \frac{3}{2 \pi^2} {,} \\
	\tilde{b}_3 &= \frac{1}{2} \int_{-1}^1 (5 x^3 - 3 x) \cos(2 \pi x) dx = 0 {,} \\
	\tilde{b}_4 &= \frac{1}{2} \int_{-1}^1 \cos(2 \pi x) dx = \frac{5}{\pi^2} - \frac{105}{8 \pi^4} {.} \\
	\end{align*}
	
	\noindent With this, we see that $p_0(x) = p_1(x) = 0$, $p_2(x) = p_3(x) = \dfrac{45}{8 \pi^2} x^2 - \dfrac{15}{8 \pi^2}$, and $p_4(x) = \left( \dfrac{45}{2 \pi^2} - \dfrac{945}{16 \pi^4} \right) \dfrac{1}{8} \left( 35 x^4 - 30 x^2 + 3 \right) + \dfrac{45}{8 \pi^2} x^2 - \dfrac{15}{8 \pi^2}$. Plots omitted.
\end{proof}

% % % % % % % % % % %
% % % Exercise 5 % % %
% % % % % % % % % % %
\begin{exercise}
	
\end{exercise}
\begin{proof}[Solution]
	
\end{proof}

% % % % % % % % % % %
% % % Exercise 6 % % %
% % % % % % % % % % %
\begin{exercise}
	Let $\phi_0(x), \phi_1(x), \phi_2(x), \ldots$ be a sequence of orthogonal polynomials on an interval $[a, b]$ with respect to a positive weight function $w(x)$. Let $x_1, \ldots, x_n$ be the $n$ zeros of $\phi_n(x)$; it is known that these roots are real and $a < x_1 < \cdots < x_n < b$.
	
	\begin{enumerate}[(a)]
		% % a % %
		\item Show that the Lagrange polynomials of degree $n - 1$ based on these points are orthogonal to each other, so we can write
		
		\[
		\int_a^b w(x) L_j(x) L_k(x) dx = 0 {,} \quad j \neq k {,}
		\]
		
		\noindent where
		
		\[
		L_j(x) = \prod_{k = 1, k \neq j}^{n} \frac{x - x_k}{x_j - x_k} {,} \quad 1 \leq j \leq n {.}
		\]
		
		\noindent [Recall Section 10.3.]
		
		% % b % %
		\item For a given function $f(x)$, let $y_k = f (x_k)$, $k = 1, \ldots, n$. Show that the polynomial $p_{n - 1}(x)$ of degree at most $n - 1$ that interpolates the function $f(x)$ at the zeros $x_1, \ldots, x_n$ of the orthogonal polynomial $\phi_n(x)$ satisfies
		
		\[
		\| p_{n - 1} \|^2 = \sum_{k = 1}^n y_k^2 \| L_k \|^2
		\]
		
		\noindent in the weighted least squares norm. This norm is defined by 
		
		\[
		\| g \|^2 = \int_a^b w(x) [g(x)]^2 dx
		\]
		
		\noindent for any suitably integrable function $g(x)$. 
	\end{enumerate}
\end{exercise}
\begin{proof}[Solution]
	\begin{enumerate}[(a)]
		% % a % % 
		\item Consider the polynomial $\phi_n(x)$. Since $\phi_n(x)$ has $n$ real roots in the interval $[a, b]$, we may express it as 
		
		\[
		\phi_n(x) = (x - x_1) (x - x_2) \cdots (x - x_n) {.}
		\]
		
		\noindent Now, notice that the product of the two Lagrange polynomials $L_j(x)$ and $L_k(x)$ may be written as 
		
		\begin{align*}
		L_j(x) L_k(x) &= \prod_{i = 1, i \neq j}^n \frac{(x - x_i)}{(x_j - x_i)} \prod_{i = 1, i \neq k}^n \frac{(x - x_i)}{(x_k - x_i)} \\
		 &= \prod_{i = 1}^n (x - x_i) \prod_{i = 1, i \neq j}^n \frac{1}{(x_j - x_i)} \prod_{i = 1, i \neq k}^n \frac{(x - x_i)}{(x_k - x_i)} \\
		 &= \phi_n(x) p_{n - 2}(x) {,}
		\end{align*}
		
		\noindent where $\phi_n(x)$ is a degree $n$ polynomial from the basis and $p_{n - 2}(x)$ is a degree $n - 2$ polynomial. $p_{n - 2}(x)$ is also just an orthogonal projection of some function $f(x)$ into the space spanned by the basis functions $\phi_0(x), \phi_1(x), \ldots, \phi_{n - 2}(x)$, and so $\phi_n(x)$ and $p_{n - 2}(x)$ are orthogonal and we have
		
		\[
		\int_a^b w(x) L_j(x) L_k(x) dx = \int_a^b w(x) \phi_n(x) p_{n - 2}(x) = 0 {,} \quad j \neq k {.}
		\]
		
		% % b % % 
		\item 
	\end{enumerate}
\end{proof}

% % % % % % % % % % %
% % % Exercise 7 % % %
% % % % % % % % % % %
\begin{exercise}
	Prove the Gram-Schmidt Theorem given on page 375. 
\end{exercise}
\begin{proof}[Solution]
	
\end{proof}

% % % % % % % % % % %
% % % Exercise 8 % % %
% % % % % % % % % % %
\begin{exercise}
	Using the recursion formula for Chebyshev polynomials, show that $T_n(x)$ can be written as
	
	\[
	T_n(x) = 2^{n - 1} (x - x_1) (x - x_2) \cdots (x - x_n) {,}
	\]
	
	\noindent where $x_i$ are the $n$ roots of $T_n$. 
\end{exercise}
\begin{proof}[Solution]
	
\end{proof}

% % % % % % % % % % %
% % % Exercise 9 % % %
% % % % % % % % % % %
\begin{exercise}
	
\end{exercise}
\begin{proof}[Solution]
	
\end{proof}

\end{document}