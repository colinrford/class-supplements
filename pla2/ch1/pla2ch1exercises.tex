\documentclass[12pt,a4]{article}

\usepackage{amsfonts,amsmath,amssymb,amsthm}
\usepackage{amscd}
\usepackage{mathtools}
\usepackage{geometry, algorithmicx} 
\usepackage[noend]{algpseudocode}
\usepackage{subfig}
\usepackage{graphicx}
\usepackage{dsfont}
\usepackage{fancyhdr}
\usepackage{enumerate}
\usepackage{hyperref}
\hypersetup{
	colorlinks,
	citecolor=black,
	filecolor=black,
	linkcolor=black,
	urlcolor=black
}
\usepackage{pgf, tikz}
\usetikzlibrary{shapes,snakes}
\usetikzlibrary{arrows, automata}
\theoremstyle{definition}


\fancyhf{}
\fancyhead[LE,RO]{\thepage}
\fancyhead[CE]{\Author}
\fancyhead[CO]{\Title}
\renewcommand\headrulewidth{0pt}
\pagestyle{fancy}

\author{Colin Ford}
\title{Complex Analysis - Chapter 1 Exercises}
\date{}

\makeatletter
\let\Title\@title
\makeatother

\newtheorem*{theorem*}{Theorem}
\newtheorem*{proposition*}{Proposition}
\newtheorem{problem}{Problem}
\newtheorem*{problem*}{Problem}
\newtheorem{exercise}{}
\newtheorem{lemma}{Lemma}
\newtheorem*{definition*}{Definition}
\newtheorem*{lemma*}{Lemma}
\newtheorem*{claim*}{Claim}
\newtheorem*{example}{Example}

\renewcommand{\Re}{\text{Re}}
\renewcommand{\Im}{\text{Im}}

\newcommand{\C}{\mathbb{C}}
\newcommand{\D}{\mathbb{D}}
\newcommand{\R}{\mathbb{R}}

\begin{document}

\maketitle

\section*{Exercises}

% % % % % % % % % % %
% % % Exercise 1 % % %
% % % % % % % % % % %
\begin{exercise}
	Describe geometrically the sets of points $z$ in the complex plane defined by the following relations:
	
	\begin{enumerate}[(a)]
		% % a % %
		\item $|z - z_1| = |z - z_2|$ where $z_1, z_2 \in \C$. 
		
		% % b % %
		\item $1 / z = \bar{z}$. 
		
		% % c % %
		\item $\Re(z) = 3$. 
		
		% % d % %
		\item $\Re(z) > c$, (resp., $\geq c$) where $c \in \R$. 
		
		% % e % %
		\item $\Re(a z + b) > 0$ where $a, b \in \C$. 
		
		% % f % %
		\item $|z| = \Re(z) + 1$. 
		
		% % g % %
		\item $\Im(z) = c$ with $c \in \R$.
	\end{enumerate}
\end{exercise}
\begin{proof}[Solution]
	\begin{enumerate}[(a)]
		% % a % %
		\item 
		
		% % b % %
		\item Rearranging this to $z \bar{z} = |z|^2 = 1$, we have the unit circle in the plane. 
		
		% % c % %
		\item This is the vertical line at $\Re(z) = 3$. Each number along this line is of the form $z = 3 + i y$.
		
		% % d % %
		\item 
		
		% % e % %
		\item 
		
		% % f % %
		\item 
		
		% % g % %
		\item 
	\end{enumerate}
\end{proof}

% % % % % % % % % % %
% % % Exercise 2 % % %
% % % % % % % % % % %
\begin{exercise}
	
\end{exercise}

% % % % % % % % % % %
% % % Exercise 3 % % %
% % % % % % % % % % %
\begin{exercise}
	
\end{exercise}

% % % % % % % % % % %
% % % Exercise 4 % % %
% % % % % % % % % % %
\begin{exercise}
	
\end{exercise}

% % % % % % % % % % %
% % % Exercise 5 % % %
% % % % % % % % % % %
\begin{exercise}
	
\end{exercise}

% % % % % % % % % % %
% % % Exercise 6 % % %
% % % % % % % % % % %
\begin{exercise}
	
\end{exercise}

% % % % % % % % % % %
% % % Exercise 7 % % %
% % % % % % % % % % %
\begin{exercise}
	
\end{exercise}

% % % % % % % % % % %
% % % Exercise 8 % % %
% % % % % % % % % % %
\begin{exercise}
	
\end{exercise}

% % % % % % % % % % %
% % % Exercise 9 % % %
% % % % % % % % % % %
\begin{exercise}
	
\end{exercise}

	% % % % % % % % % % %
	% % % Exercise 10 % % %
	% % % % % % % % % % %
	\begin{exercise}\label{ex:10}
		Show that
	
		\[
		4 \frac{\partial}{\partial z} \frac{\partial}{\partial \bar{z}} = 4 \frac{\partial}{\partial \bar{z}} \frac{\partial}{\partial z} = \triangle {,}
		\]
	
		\noindent where $\triangle$ is the \textbf{Laplacian}
	
		\[
		\triangle = \frac{\partial^2}{\partial x^2} + \frac{\partial^2}{\partial y^2} {.}
		\]
	\end{exercise}
	\begin{proof}
		Recall that 
		
		\[
		\frac{\partial}{\partial z} = \frac{1}{2} \left( \frac{\partial}{\partial x} + \frac{1}{i} \frac{\partial}{\partial y} \right) \quad \text{and} \quad \frac{\partial}{\partial \bar{z}} = \frac{1}{2} \left( \frac{\partial}{\partial x} - \frac{1}{i} \frac{\partial}{\partial y} \right) {.}
		\]
		
		\noindent $\triangle$ is obtained by a simple computation of these two operators. Indeed, 
		
		\begin{align*}
		\frac{\partial}{\partial z} \frac{\partial}{\partial \bar{z}} &= \frac{1}{2} \left( \frac{\partial}{\partial x} + \frac{1}{i} \frac{\partial}{\partial y} \right) \cdot \frac{1}{2} \left( \frac{\partial}{\partial x} - \frac{1}{i} \frac{\partial}{\partial y} \right) \\
		 &= \frac{1}{4} \left( \frac{\partial^2}{\partial x^2} - \frac{1}{i} \frac{\partial^2}{\partial x \partial y} + \frac{1}{i} \frac{\partial^2}{\partial y \partial x} + \frac{\partial^2}{\partial y^2} \right) \\
		 &= \frac{1}{4} \left( \frac{\partial^2}{\partial x^2} + \frac{\partial^2}{\partial y^2} \right) \\
		 &= \frac{1}{4} \triangle {,} 
		\end{align*}
		
		\noindent where we have used the following fact
		
		\[
		\frac{\partial^2}{\partial x \partial y} = \frac{\partial^2}{\partial y \partial x} {,}
		\]
		
		\noindent a consequence of Fubini's theorem. The other $\frac{\partial}{\partial \bar{z}} \frac{\partial}{\partial z}$
		

	\end{proof}

	% % % % % % % % % % %
	% % % Exercise 11 % % %
	% % % % % % % % % % %
	\begin{exercise}
		Use Exercise 10 to prove that if $f$ is holomorphic in the open set $\Omega$, then the real and imaginary parts of $f$ are \text{harmonic}; that is, their Laplacian is zero. 
	\end{exercise}
	\begin{proof}
		Suppose $f$ is holomorphic in an open set $\Omega$. (later; application of Exercise \hyperref[ex:10]{\ref{ex:10}} and the Cauchy-Riemann equations).
	\end{proof}

% % % % % % % % % % %
% % % Exercise 12 % % %
% % % % % % % % % % %
\begin{exercise}
	
\end{exercise}

% % % % % % % % % % %
% % % Exercise 13 % % %
% % % % % % % % % % %
\begin{exercise}
	
\end{exercise}

% % % % % % % % % % %
% % % Exercise 14 % % %
% % % % % % % % % % %
\begin{exercise}
	
\end{exercise}

% % % % % % % % % % %
% % % Exercise 15 % % %
% % % % % % % % % % %
\begin{exercise}
	
\end{exercise}

% % % % % % % % % % %
% % % Exercise 16 % % %
% % % % % % % % % % %
\begin{exercise}
	
\end{exercise}

% % % % % % % % % % %
% % % Exercise 17 % % %
% % % % % % % % % % %
\begin{exercise}
	
\end{exercise}

\end{document}